\documentclass[a4paper,11pt]{article}
\usepackage{a4wide}
\usepackage{t1enc,lmodern,enumitem}
\usepackage{amsmath,amsthm,amssymb}
\usepackage[utf8]{inputenc}
\usepackage[english]{babel}
\usepackage[separate]{solutions}

%--------------------------------------------
\hbadness=10000
\emergencystretch=\hsize
\tolerance=9999
\textheight=9.0in
%--------------------------------------------

\def\R{{\mathbb R}}

\begin{document}
\begin{center}
  {\large \bf Testing the \texttt{solutions} package with \texttt{separate} option.}
\end{center}
\vspace{1em}

\begin{exercise}
  First exercise using a standard layout.
  \begin{questions}
    \item First question.
    \solution{A short answer.}
    \item Second question.
    \solution{A somehow more realistic answer ending with an equation
    \begin{align*}
      \forall \, x,y \, \in \, \R, \, (x + y)^2 = x^2 + y^2 + 2xy. \endhere
    \end{align*}}
  \end{questions}
  Let us introduce some more material before adding new questions.
  \begin{questions}[resume]
    \item A third question.
    \solution{And its solution.}
  \end{questions}
\end{exercise}

\begin{exercise}
  Soit $f_1, \dots, f_N$ $N$ fonctions de $\R_+$ dans $\R$ et $G_1, \dots, G_N$
  $N$ gaussiennes centrées réduites et indépendantes. Montrer que le processus
  $X$ défini par $X_t = \sum_{i=1}^N f_i(t) G_i$ est un processus gaussien et
  calculer ses fonctions de moyenne et de covariance.  
  
  \solution{Il faut montrer que pour tous $t_1, \dots, t_n$, le vecteur $\bar
    X^n = (X_{t_1},\dots, X_{t_n})$ est un vecteur gaussien. Soit $a \in R^n$, 
    \begin{align*}
      a \cdot \bar X^n & = \sum_{i=1}^n a_i X_{t_i} = \sum_{i=1}^n a_i
      \sum_{j=1}^N f_j(t_i) G_j 
      = \sum_{j=1}^N \left(\sum_{i=1}^n a_i f_j(t_i)\right) G_j
    \end{align*}
    On conclut puisque  le vecteur $G_1, \dots, G_N$ est un vecteur gaussien. 
  }
\end{exercise}


\pagebreak
\end{document}
